\subsection{\en{Fundamentals of MRI (Imaging - ROIs)}}
\en{
A big part of studying, understanding and diagnosing neurodegenerative diseases are brain imaging techniques, such as Magnetic Resonance Imaging (MRI), Positron Emission Tomography (PET), Computerized Tomography (CT) and others. This work focuses on data collected with the method of MRI, which were used to recognize signs of MCI and AD. 

The MRI technique works by measuring the energy signal of (typically) hydrogen nuclei, as a result of excitation by external radio frequency pulses. The MRI technique is frequently split into two different processes, based on the decay of the RF-induced  nuclear magnetic resonance spin polarization, named T1 and T2, each producing different results depending on the signal and the tissue being imaged. Depending on the parameters of the process being used, MRI can imprint pictures of the anatomy of the human body, as well as its physiological processes. \cite{27}

Oftentimes, to avoid examining every single point of space (commonly referred as a 'voxel' or volume-pixel), the scan is segmented in specified Regions Of Interest (ROIs). These regions are produced by segmenting the original image, either automatically, using Machine Learning methods or by employing previously computer computed brain atlases. Specifying however, the aforementioned regions, is quite complex, since there is a great variability of neuroanatomy between humans. It is believed that the brain's function is associated with structural and functional connectivities, and therefore identifying standardized and reliable is crucial for understanding the connection between the architecture of the brain and its function.  (\cite{28}, \cite{29}, \cite{30})

Because the MRI method produces a tremendous amount of data for each scan, frequently before the main task, preprocessing is applied to capture the desirable information, while maintaining ease of data manipulation. Such methods include data augmentations, feature selection, dimensionality reduction, etc.
}