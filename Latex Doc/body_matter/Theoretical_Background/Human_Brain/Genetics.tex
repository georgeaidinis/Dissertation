\subsection{\en{Fundamentals of genetics (SNPs)}}
\en{DNA in humans is arranged in chromosome pairs, with each cell having in its nucleus 23 pairs, 22 of which are autosomes and 1 pair being the sex chromosomes. In each of the 22 pairs of chromosomes, DNA is stored in identical copies, with specific chunks being characterized as genes. A gene contains genetic information in the form of long sequences of nucleotides (A,T,G,C). 

If a change or variation in one or more nucleotide positions of a chromosome in the DNA sequence is found, it is called a Single Nucleotide Polymorphism (SNP - pronounced "snip") . If most humans have a specific certain nucleotide in a specific position of the genome, and a SNP occurs in some individuals in that exact position, then this position is said to have more than one allele. Because the DNA is stored in pairs of chromosomes, a person's DNA can contain in a specific position of the genome one SNP, two SNPs, or no SNPs at all. (\cite{31}, \cite{32})

Observed SNPs can be associated with a disease, however, it may not always directly be the cause for that disease. An example of this is the APOE gene (chromosome 19 position q13.32), which has been determined to be a risk factor for AD, specifically the ε4 allele. There are three versions of the gene in humans, ε2, ε3, and ε4, with ε3 being the most prevalent, the existence of the ε4 variant being a risk factor, while having two ε2 alleles being associated with lower probabilities of developing the disease. The disease however is also associated with other gene mutations, such as mutations in the genes APP, PSEN1, PSEN2 and others, which especially influence the early onset variant of the disease.  (\cite{32}, \cite{33}, \cite{34})

Recognizing the different SNPs that are contained in the human genome is the
topic of studies such as Genome Wide Association Studies (GWAS), which are a
part of the field of bioinformatics. Understanding the changes in the genome
can help recognize how they translate in the phenotype, help with their
treatment and their prevention. (\cite{35})
}
