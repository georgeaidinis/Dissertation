\chapter{\tl{Theoretical Background}}
	
\section{\tl{Human Brain}}
\subsection{\en{Aging}}
\en{As the human body ages, all organs experience age-related effects, and so does the brain. Both physiologically and cognitively, there are several changes that can be observed as part of the normal brain aging process. \cite{12}

Cognitively, memory (specifically episodic and semantic memory) is one of the core areas that are affected. Older people may be forgetting names of items or persons, having to repeat questions, misplacing items, getting lost, and having trouble recalling information in general. Additionally, language skills such as vocabulary and language skills may be affected, as well as the ability to learn new skills and multitask. \cite{13}

Physiologically, the brain shrinks in the areas of the frontal lobe as well as the hippocampus, the areas that are generally thought to be linked with higher cognitive function and memory, at a rate of 5\% per decade after the age of 40. This effect is due to the grey matter shrinkage, which is attributable to neuronal cell death. Additionally, cortical density decline is observed, meaning that the outer surface of the brain is becoming thinner. This effect is more pronounced in the frontal and temporal lobes. White matter also declines, with the myelinating regions of the frontal lobe being most affected by white matter lesions. Finally, the levels of neurotransmitters such as dopamine and serotonin see a steep decrease, an effect that has been associated with declines in cognitive and motor performance.  (\cite{14}, \cite{15})

While the aforementioned symptoms are very much similar to the symptoms that Mild Cognitive Impairment and Alzheimer's Disease exhibit, normal aging patterns of decline are divergent from the ones of MCI and AD. The effects of normal brain aging are characterized by their occasional nature, while MCI's and AD's ones are more consistent, and gradually worsen in some cases, and they are accompanied by other dementia symptoms, such as confusion, mood changes and others. Furthermore, the physiological changes of the brain are much more pronounced and significantly more severe. MCI and AD have a much more noticeable effect on the person's daily life, and some cases need assistance in order to perform normal daily tasks. \cite{16}
}