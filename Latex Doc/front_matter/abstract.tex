\begin{abstract}
Η νόσος Αλτσχάιμερ αποτελεί αντικείμενο ολοένα και περισσότερων μελετών, αφού αποτελεί μια από τις σημαντικότερες νευροεκφυλιστικές ασθένειες. Η χρήση υπολογιστικών μεθόδων για την διάγνωση, την μελέτη αλλά και την αντιμετώπιση γνωρίζει ραγδαία ανάπτυξη, και εφαρμόζονται ως επί το πλείστον μέθοδοι μηχανικής μάθησης για την αποτύπωση των δεδομένων, την επεξεργασία και τον μετασχηματισμό τους, αλλά και την κατηγοριοποίηση τους. Οι σύγχρονες μέθοδοι χρησιμοποιούν πολυτροπικά δεδομένα, με την έμφαση να δίνεται στα απεικονιστικά και στα γενετικά δεδομένα. Στο πλαίσιο της παρούσας διπλωματικής εργασίας διερευνούνται εκτενώς διάφορες μέθοδοι ανάλυσης δεδομένων, μηχανικής μάθησης αλλά και βαθειάς νευρωνικής μάθησης, καθώς και οι μεταξύ τους συνδυασμοί. Το πρόβλημα που μελετάται είναι αυτό της κατηγοριοποίησης δεδομένων από το σύνολο δεδομένων \en{Alzheimer's Disease Neuroimaging Initiative} σε πάσχοντες από νόσο του Αλτσχάιμερ, άτομα με ήπια νοητική διαταραχή, και φυσιολογικά. Το σύνολο δεδομένων περιέχει απεικονιστικά αλλά και γενετικά δεδομένα από 1567 συμμετέχοντες. Οι μέθοδοι ανάλυσης δεδομένων που εξετάζονται είναι οι \tl{Deep Canonical Correlation Analysis, Multiple Correspondence Analysis, Orthonormal Projective Non-Negative Matrix Factorisation} και \tl{Factor Analysis of Mixed Data}. Οι μέθοδοι που χρησιμοποιήθηκαν για την κατηγοριοποίηση είναι τα \tl{Support Vector Machines} καθώς και μέθοδοι \tl{Ensemble Classifiers}. Για κάθε πιθανό συνδυασμό, τα μοντέλα αυτά αξιολογήθηκαν ως προς την απλή ακρίβειά τους, το \tl{F1 Score}, και την εξισορροπημένη ακρίβειά τους. Παρατίθονται τα αποτελέσματα, σχολιασμός των συγκρίσεων, συμπεράσματα καθώς και μελλοντικές επεκτάσεις. 

   \begin{keywords}
   Νόσος Αλτσχάιμερ, Ήπια Νοητική Διαταραχή, Βαθειά Νευρωνική Μάθηση, Μηχανική Μάθηση,     \tl{Data Analysis, Classification, Deep Canonical Correlation Analysis, Non-Negative Matrix Factorization, Correspondence Analysis}
   \end{keywords}
\end{abstract}



\begin{abstracteng}
\en{
Alzheimer's Disease (AD) is subject to an increasing number of studies, since it is one of the most important neurodegenerative diseases. The use of computational methods for the diagnosis, studying and treatment of the disease has enjoyed rapid growth, while presently, mostly machine learning are applied for the visualization, processing, transformation and classification of the data related to the disease. Modern methods utilize multi modal data, with the focus being on imaging and genetic modals. In this study, a multitude of data analysis, Machine and deep learning methods are extensively studied, as well as the combinations thereof. The subject of the task is that of classifying data from the Alzheimer's Disease Neuroimaging Initiative dataset into AD patients, mild cognitive impairment patients, and cognitive normal people. The dataset contains imaging as well as genetic data from 1567 participants. The Data Analysis methods that were studied were those of Deep Canonical Correlation Analysis, Multiple Correspondence Analysis, Orthonormal Projective Non-Negative Matrix Factorisation and Factor Analysis of Mixed Data. The classification methods that were used were those of Support Vector Machines, and Ensemble Classifier methods. For every combination of the aforementioned methods, the models were evaluated on their accuracy, their F1 Score and their balanced accuracy. The results of the study are presented, commented on, conclusions are drawn and future directions are discussed. 
}
   \begin{keywordseng}
    \tl{Alzheimer's Disease, Mild Cognitive Impairment, Deep Learning, Machine Learning, Data Analysis, Classification, Deep Canonical Correlation Analysis, Non-Negative Matrix Factorization, Correspondence Analysis}
   \end{keywordseng}

\end{abstracteng}