\subsection{\en{Mild Cognitive Impairment}}
\en{Mild Cognitive Impairment is a state of a person that is characterized by problems with memory, language, thinking or judgment. It is usually observed between the stage of normal cognitive decline that happens to humans due to aging and the dramatic fall in cognition that is apparent to people with dementia. People with MCI have memory loss or other cognitive ability loss, exceeding the normal decline due to age and are not demented. \cite{17}

MCI’s symptoms can manifest in many different functions of the human brain, including weaker memory, poor reasoning and judgment skills, visual perception and others. Frequently, MCI coexists with other illnesses or emotions, such as depression, anxiety, irritability and aggression, or apathy. \cite{18}

The cause of the disorder is unknown; it is believed that MCI is caused by the same mechanisms that are thought to be responsible for the neuropathology of the early stages of Alzheimer’s Disease, however that is unproven. Risk factors include age, family history of AD or dementia, genetic factors, and other medical conditions such as diabetes, high blood pressure, smoking, obesity, depression etc. \cite{18}

People with diagnosed MCI have a significantly higher chance than that of cognitive normal population to develop AD or some form of dementia. Despite the fact that there is no standardized test for MCI, clinical characterization is achieved through the results of various tests (such as mental tests, neurological exams, patient family history, brain imaging and searching for biomarkers) and the informations that the patient provides. \cite{19}

It is not exactly clear how to prevent MCI, but studies show that engaging in frequent physical activity, maintaining a healthy and balanced diet, engaging socially with others, being mentally active, reducing alcohol and not smoking may be mitigating factors to the risk of developing the condition. \cite{19}}
