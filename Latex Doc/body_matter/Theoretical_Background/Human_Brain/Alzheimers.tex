\subsection{\en{Alzheimer's Disease}}
\en{Alzheimer’s Disease is a neurodegenerative disease that affects the brain both biologically and cognitively. It was first reported by A. Alzheimer in 1906, but described as a disease only after 1910, by E. Kraepelin. AD is the most common cause of dementia, a term used to describe a group of symptoms that include decline in memory, reasoning, and other thinking skills. It is reported that AD is responsible for about 70\% of dementia cases. (\cite{20},\cite{21},\cite{22})

AD is a progressive condition, meaning the symptoms gradually appear and worsen over time. Early symptoms include short term memory loss, decline in conversational abilities and poor reasoning. As the disease progresses, patients have trouble recalling names, may have confusion and obsessive, repetitive or impulsive behaviour, serious problems with speaking and the use of language, and generally problems that require external assistance in their daily life. In later stages of the disease, the patients have trouble with even the most basic tasks, such as eating and moving, and require full time assistance. Gradually, the condition of the patients deteriorates, ultimately leading to death. \cite{23}

The cause of AD is unknown, but genetic and environmental risk factors have been implicated. AD is linked with the formation and buildup of plaques (abnormal clusters of protein fragments) of the protein amyloid β, and neurofibrillary tangles (twisted strands of protein) of the tau protein. There are several hypotheses as to the disease’s origin, yet none of them have been confirmed. There are two perhaps significant hypotheses, the amyloid and the cholinergic Hypothesis.  \cite{24}

AD is a multifactorial disease, being associated with several risk factors, such as age and gender, genetic factors, life style, coexistent or previous diseases, head injuries and environmental factors. The most important however is age, with most AD cases having a late onset that starts after 65 years of age. Normal brain aging is characterized by a reduction in brain volume and weight, a loss of synapses, and the enlargement of ventricles. These changes appear in AD patients as well, but more profound in general. There are two categories of AD based on the age that it appears, Early Onset AD which is generally familial and displays inheritance and has onset age that ranges from 30 to 60 years of age (1-6\% of cases), and late onset AD, which is by far more common and has age of onset above 65 years. Genetic factors also play a significant role, with 70\% of AD cases being related to genetic factors. The genes APP, PSEN-1, PSEN-2 and most importantly ApoE are associated with AD. (\cite{24}, \cite{25})

In order to successfully diagnose the disease, a practitioner has to evaluate the person, with multiple tests if necessary. The diagnosis is based on medical history, advanced medical imaging of the brain (using CT or MRI or PET or SPECT), mental tests such as the MMSE, blood tests and psychological tests for depression, since depression can either be concurrent with Alzheimer's disease, an early sign of cognitive impairment, or even the cause. (\cite{25}, \cite{26})

There's currently no cure for Alzheimer's disease, however there are certain medications available that can temporarily mitigate the symptoms. Since the cause of the disease is still unknown, there is no designated preventive roadmap. Despite that, frequent physical exercise, a healthy and balanced diet as well as staying mentally and socially active all have been linked to lower rates of AD. 
}