\chapter{\tl{Conclusions}} 
\en{
This study was intended to be an extensive comparison of the applications of data analysis methods, as well as machine and deep learning methods, applied to the problem of CN / MCI / AD classification. From the work that was performed, there are some clear messages. First of all, data analysis techniques such as MCA and OPNMF are better compared to the MUSE ROIs because the OPNMF components are data driven while MUSE ROIs are derived from templates. MCA is also better than the vanilla genetic view, because it transforms the genetic data into the type of the imaging data, making both views have the same kind. Furthermore, the method of Deep Canonical Correlation Analysis as stated in the original paper, is not beneficial to this problem, at least not without further tuning, however it increases the correlation between the views in agreement with the original paper. ensemble classifier methods are superior to the simplistic single classifier methods such as Support Vector Machines. Finally, it is clear that using only genetic data is not sufficient to yield higher-quality results, as they only show predesposition for the disease. The combinations that achieved the best results used imaging or imaging along with genetic data, either OPNMF-transformed imaging data or MCA-transformed genetic data, accompanied with either Bagging ensembles of SVMs or a polynomial kernel SVM as classifiers. 
}