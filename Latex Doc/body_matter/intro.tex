\chapter{\tl{Introduction}}
\en{
A rapidly growing cause of death in the developed countries are neurodegenerative diseases. Diseases such as the Alzheimer's, as well as disorders such as Mild Cognitive Impairment are most prevalent on old people, with ages equal or older than 65 years old, being the group that's most affected by them. It is estimated that in the US alone, 6.5 million people are suffering from Alzheimer’s Disease today, while globally this number can be as high as 35 million. Those estimations are expected to grow to 135 million globally, and many of them are undiagnosed and or untreated even today.(\cite{1},\cite{2})

During the normal course of aging, the human brain displays changes, both anatomical as well as functional, that seem to be accelerated in patients with such diseases. The human brain is developed until the age of 25 years, and after that it continuously loses neural mass, leading to brain atrophy, a condition that is studied extensively, thanks to advances the field of medical imaging. The atrophy effect is sped up in some patients quite significantly, and in the case of the AD, it leads to neuronal decay and loss, and eventually death.

The effects of AD vary quite significantly, and are apparent not only clinically, but in imaging scans and genetic surveys as well. Because of low awareness, MCI and AD are often mistakenly associated with getting older, while due to the need of an experienced practitioner for the diagnosis, many cases go undiagnosed. However, a great deal of research is being done on studying the symptoms, potential causes, as well as the treatment of the disease, with some estimates putting the total cost of Alzheimer’s research in the tens of billions USD. Furthermore, for the total cost for the healthcare related to AD patients for the year 2020 in the US has been estimated to be around 300 billion USD. (\cite{2}, \cite{3}).

Despite the considerable resources, no clear cause has been found for AD, no definitive prevention method has been found, and no treatment method has been widely successful. 
}
\section{\en{Contents of this Thesis}}
\en{
To tackle the problem of diagnosing Alzheimer’s as well as MCI, and also predicting and modelling their respective courses, a plethora of studies have been published. A common characteristic that many have is the use of machine and deep learning methods, especially on neuroimaging, genetic and clinical data collected from patients of the diseases. The main drivers behind this effort are the advances in medical imaging and the staggering growth of computational abilities in recent years, making more complex and better methods applicable and practical. (\cite{4}, \cite{5})

Most studies employ neuroimaging data collected with the Magnetic Resonance Imaging and Positron Emission Tomography methods, that were collected from the AD neuroimaging Initiative database. The imaging data are often accompanied with biomarkers, as well as Single Nucleotide Polymorphisms (SNPs), and even data from clinical tests performed by licensed practitioners. (\cite{6}, \cite{7}, \cite{8}).

While in general more complex models employing different modalities of the data performed better, there is no unified approach, and the diversity of models is intriguing. However, most studies focus on either classifying whether or not a subject is a patient of AD and predicting the course of the disease, while some papers focus on predicting if a patient with MCI will convert to Alzheimer’s. \cite{4}

A common problem most studies had to overcome was the overabundance of features that the data had, while having too few samples. This is known as the ‘curse of dimensionality’ problem, and is associated with biomedical data. The data is characterised by very large dimensions (especially in the case of the neuroimaging view), yet not enough samples. This is due to the imaging techniques' post-collection data processing, resulting in very high dimensions, while the sample size is very low, due to the difficult, sometimes inaccessible, expensive and long-lasting nature of the technique. This has the adverse consequence of data not being easily visualised (since they cannot be interpreted in the three dimensional space humans are familiar with), along with the tendency of the models to overtrain and overfit on the low number of datapoints. To avoid the aforementioned problems, data analysis techniques have been used, with them being as simple as PCA, or as complex as employing deep learning, for example neural networks. (\cite{9}, \cite{10}, \cite{11})

This study attempts to create a comparative analysis of machine learning methods and algorithms being applied to the problem of predicting whether a subject is cognitive normal, has MCI, or AD. The data was collected through the ADNI dataset, and more specifically, imaging data taken from T1-weighted MRI scans, along with genetic data in the form of SNPs. 

In this thesis, we also explore data analysis techniques in order to tackle the dimensionality curse problem, as well as transformations and statistical analysis methods. To convert the exceedingly dissimilar imaging and genetic views, we experimented with a novel technique, called Deep Canonical Correlation Analysis, where neural networks are used to learn a transformation of the different views’ features into a hyperspace that is better linearly correlated than the raw data, and thus potentially easier to perform the classification task.

Furthermore, we experimented with dimensionality reduction methods, such as Multiple Correspondence analysis, Orthonormal Projective Non-Negative Matrix Factorization, and Factor Analysis of Mixed Data. All of the possible combinations of the techniques were used, and their effect on the classification task was compared.  

For the classification task in particular, we present the results from applying Support Vector Machines, which are widely regarded as a fairly simple, understandable, yet practical and capable model, as well as Ensemble Learning methods such as Bagging and Adaboost to the problem. For the ensemble methods, we experimented with Decision Trees and again Support Vector Machines as base model classifiers, and the yielded results were compared not only between them, but also to the single-classifier results. 
}
\section{\en{Structure of the Thesis}}
\en{
This work is divided into 7 chapters, including the introduction chapter. Chapter 2 introduces some theoretical knowledge in order to tackle the problem described in the introductory chapter, pertaining to the human brain and some fundamentals about the machine and deep learning methods that were used. Chapter 3 explains the methodology that was used, with a brief description of the dataset that was used, the parameters of the methods, as well as the metrics to evaluate them. On chapter 4, we present some results from the attempt at optimizing the models that were later used for the classification tasks, and on chapter 5 the classification results are presented. Finally, on chapters 6 and 7, conclusions are drawn from observations made on the results, future extensions and directions are discussed, as well as the practical limitations of this work.
}